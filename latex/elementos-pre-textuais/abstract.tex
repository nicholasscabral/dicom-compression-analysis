This study investigates the relationship between compression rates and information loss in \acrshort{DICOM}-standard medical tomography images. Three compression methods were evaluated: \acrshort{PNG}, \acrshort{JPEG}, and \acrshort{PCA}. The objective is to highlight the scenarios where each method stands out, balancing image size reduction and the preservation of diagnostic quality. \acrshort{PNG} was analyzed for its lossless compression capability, ensuring complete preservation of image quality. \acrshort{JPEG} was studied as a lossy compression method, suitable for significant size reduction while maintaining acceptable visual quality. \acrshort{PCA} explored dimensionality reduction by adjusting compression levels based on the percentage of explained variance, allowing flexibility between compression and information retention. The comparative analysis revealed the advantages and limitations of each method, highlighting how image characteristics (e.g., homogeneous areas or fine details) influence compression results. This study contributes to the selection of efficient compression strategies in hospital systems, optimizing the storage of medical images in the \acrshort{DICOM} standard.

\textbf{Keywords:} Data Compression. Medical Images. \acrshort{DICOM}. Visual Quality. Dimensionality Reduction. Computed Tomography.
