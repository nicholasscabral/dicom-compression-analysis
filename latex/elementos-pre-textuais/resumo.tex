Este estudo investiga a relação entre taxas de compressão e perda de informações em imagens médicas de tomografia no padrão \acrshort{DICOM}. Foram utilizados três métodos de compressão: \acrshort{PNG}, \acrshort{JPEG} e \acrshort{PCA}. O objetivo é evidenciar os cenários onde cada método se destaca, conciliando a redução do tamanho das imagens com a preservação da qualidade necessária para fins diagnósticos. O \acrshort{PNG} foi avaliado por sua capacidade de compressão sem perdas, garantindo a preservação completa da qualidade visual da imagem. O \acrshort{JPEG} foi analisado como uma alternativa de compressão com perdas controláveis, ideal para redução significativa de tamanho com comprometimento aceitável da qualidade. Já o \acrshort{PCA} explorou a redução de dimensionalidade, utilizando percentuais de variância explicada para ajustar a compressão conforme a necessidade de retenção de informações. A análise comparativa revelou vantagens e limitações de cada método, destacando como o conteúdo das imagens (como áreas homogêneas ou detalhes finos) influencia os resultados de compressão e qualidade visual. Este estudo contribui para a escolha de estratégias adequadas de compressão em sistemas hospitalares, otimizando o armazenamento de imagens médicas no padrão \acrshort{DICOM}.

\textbf{Palavras-chave}: Compressão de Dados. Imagens Médicas. \acrshort{DICOM}. Qualidade Visual. Redução de Dimensionalidade. Tomografia Computadorizada.
