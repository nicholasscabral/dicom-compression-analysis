\chapter{Conclusão}
\label{cap:conclusao}

Neste trabalho, foi realizada uma análise comparativa de três métodos de compressão de imagens médicas no formato \acrshort{DICOM}: \acrshort{PNG}, \acrshort{JPEG} e \acrshort{PCA}. O estudo investigou o impacto desses algoritmos na redução do tamanho dos arquivos e na preservação da qualidade das imagens médicas, considerando três tipos de órgãos: pulmão, mama e cérebro. Os resultados demonstraram diferenças significativas entre os métodos, evidenciando vantagens e limitações que dependem tanto das características dos algoritmos quanto do conteúdo das imagens.

As questões de pesquisa levantadas foram abordadas de maneira detalhada. Primeiramente, quanto às taxas médias de compressão alcançadas pelos algoritmos, o \acrshort{JPEG} obteve os melhores resultados, superando 96\% em todos os órgãos analisados. O \acrshort{PNG} apresentou taxas moderadas e consistentes de 83-86\%, conciliando uma alta taxa de compressão com preservação total da qualidade. Já o \acrshort{PCA} demonstrou um desempenho variável, dependendo do percentual de variância explicada, sendo mais eficiente em cenários onde a retenção parcial da informação é aceitável.

Sobre o impacto das características das imagens de cada órgão nas taxas de compressão, foi constatado que imagens de pulmão, devido às suas grandes áreas homogêneas, favoreceram a compressão em todos os algoritmos, resultando nas melhores taxas. Por outro lado, as imagens de mama, com maior densidade de detalhes, apresentaram maior resistência à compressão. As imagens de cérebro, por sua vez, exibiram comportamento intermediário, refletindo sua composição equilibrada entre áreas homogêneas e detalhadas.

Em relação ao impacto da variação da taxa de compressão na qualidade visual, os valores de \acrshort{PSNR} demonstraram que o \acrshort{PNG} manteve a qualidade perfeita das imagens, com \acrshort{PSNR} tendendo ao infinito. O \acrshort{JPEG} apresentou valores entre 42 e 45 dB, indicando alta qualidade visual com perdas imperceptíveis a olho nu. Já o \acrshort{PCA}, dependendo do percentual de variância explicada, teve qualidade mais variável, com melhores resultados observados em níveis mais altos (99\%).

Finalmente, no que diz respeito à escolha do algoritmo ideal para cada cenário, conclui-se que o \acrshort{JPEG} oferece o melhor equilíbrio entre compressão e qualidade visual, sendo uma opção eficiente para armazenamento em larga escala. O \acrshort{PNG} é recomendado para aplicações que exigem preservação total da qualidade, enquanto o \acrshort{PCA} é mais indicado para cenários onde a flexibilidade na retenção de dados é prioritária, apesar de seu maior custo computacional.

Este estudo contribui para o entendimento das particularidades de cada método de compressão e oferece subsídios para a escolha de estratégias adequadas em diferentes cenários hospitalares. Como trabalhos futuros, sugere-se:
\begin{itemize}
    \item Avaliar outros algoritmos de compressão, como \acrshort{AVIF} e \acrshort{SPIHT}, que também são amplamente utilizados em imagens médicas e científicas.
    \item Explorar métodos baseados em aprendizado de máquina, como compressão neural.
    \item Investigar o impacto da compressão na acurácia de diagnósticos realizados a partir dessas imagens com o auxílio de um profissional da saúde especializado em diagnóstico de imagens médicas.
    \item Ampliar o estudo para outros tipos de exames médicos, como ressonância magnética e ultrassonografia, a fim de verificar se os resultados observados neste trabalho se mantêm em outros contextos.
\end{itemize}

Com base nos resultados obtidos, conclui-se que a compressão de imagens médicas, quando bem planejada, pode oferecer soluções eficazes para os desafios de armazenamento em ambientes hospitalares, conciliando redução de tamanho e preservação da integridade das informações essenciais ao diagnóstico.

