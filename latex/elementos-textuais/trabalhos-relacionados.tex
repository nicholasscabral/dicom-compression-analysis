\chapter{Trabalhos Relacionados}
\label{cap:relac}

Neste capítulo, apresentamos uma revisão dos trabalhos relacionados ao tema do nosso estudo. O objetivo é fornecer uma visão geral dos estudos anteriores que abordam questões semelhantes às que estamos investigando. Ao revisar esses trabalhos, buscamos identificar lacunas no conhecimento existente e destacar como nossa pesquisa contribui para preencher essas lacunas.

\section{Processos de Compressão de Dados Aplicados a Imagens Médicas} 
No estudo \citeonline{stolfi2000MedicalCompression}, o autor aborda a aplicação de técnicas de compressão de dados em imagens médicas, com foco específico na cineangiografia, uma técnica de imagem que registra o fluxo sanguíneo dentro dos vasos. Ele ressalta a importância da qualidade e fidelidade na reprodução das imagens médicas para diagnósticos precisos e procedimentos médicos. Reconhecendo a necessidade de altas taxas de compressão sem perdas, o autor destaca os desafios associados especialmente em imagens ruidosas.

Para abordar esses desafios, propõe-se a implementação de um sistema de compressão de imagem em movimento, com ênfase na cineangiografia. O autor desenvolve um processo de compensação de movimento baseado em semelhança de forma para otimizar a extração de vetores em movimento, visando melhorar tanto a taxa de compressão quanto a qualidade da imagem reconstruída em comparação com métodos tradicionais. Além disso, explora-se o uso de quantização não linear e pós-processamento das imagens reconstruídas para reduzir a visibilidade de erros introduzidos durante o processo de compressão. Destaca-se a importância da avaliação dos resultados por meio de um ambiente de visualização de imagens, permitindo comparações entre diferentes métodos de processamento.

Em resumo, o trabalho representa uma contribuição significativa para a área de compressão de dados em imagens médicas, uma vez que explora essas compressões para o contexto específico  da cineangiografia. 

\section{Compressão de Imagens Mamográficas Utilizando Segmentação e o Algoritmo \acrshort{PPM}}
A principal problemática abordada no artigo de \citeonline{marques2007mamografia} é a necessidade de  comprimir imagens mamográficas de forma eficiente, preservando informações essenciais para diagnósticos precisos. A digitalização de imagens mamográficas resulta em arquivos de grande tamanho, o que pode sobrecarregar os sistemas de armazenamento e dificultar a transmissão dessas imagens, impactando a eficiência dos processos médicos. Para solucionar essa questão, os autores propuseram um método que combina a segmentação por limiarização, a decomposição da imagem em planos de bits e a utilização do algoritmo de compressão \acrfull{PPM}. Essa abordagem permitiu alcançar taxas de compressão competitivas em relação a outros compressores avançados, ao mesmo tempo em que preservou a qualidade e a integridade das informações contidas na imagem original.

Em resumo, o estudo destaca a relevância da inovação tecnológica na área de compressão de imagens médicas, demonstrando  que a combinação dessas técnicas pode resultar em benefícios significativos para a área da saúde. Essa abordagem não apenas contribui para a otimização do armazenamento e transmissão de imagens mamográficas, mas também pode facilitar a análise e o compartilhamento desses dados, auxiliando profissionais de saúde em diagnósticos mais precisos e eficazes. 

% Meu estudo visa expandir as métricas eficiência de compressão de dados para imagens mamográficas, uma  vez que utiliza métodos de compressão modernos.


\section{Utilização da Análise de Componentes Principais na compressão de imagens digitais}
A problemática abordada por \citeonline{santos2012PCACompression} estava relacionada à necessidade de comprimir imagens médicas de forma eficiente, mantendo as informações essenciais para o diagnóstico clínico, ao mesmo tempo em que se reduz o espaço de armazenamento necessário. A compressão de imagens é crucial em ambientes médicos, onde grandes volumes de dados de imagem são gerados diariamente e o armazenamento eficiente é fundamental para a gestão e análise dessas informações. O autor explorou o \acrshort{PCA} como uma ferramenta estatística para redução da dimensionalidade dos dados de imagem. A aplicação do \acrshort{PCA} permitiu representar as imagens em uma estrutura computacional de dimensão reduzida, preservando as características principais da imagem original. Isso foi alcançado por meio da projeção dos dados em um subespaço gerado por um sistema de eixos ortogonais, obtido através do método \acrfull{SVD} \cite{santos2012PCACompression}.

Os resultados do estudo demonstraram que as imagens médicas comprimidas mantiveram suas principais características até aproximadamente $1/4$ do tamanho original, ou seja, uma taxa de compressão de aproximadamente 75\%, evidenciando a eficácia do \acrshort{PCA} como uma ferramenta de compressão de imagens. A quantidade de componentes principais utilizadas na compressão foi destacada como um fator crucial, pois influencia diretamente na recuperação da imagem original a partir da imagem compactada. A taxa de compressão obtida foi um aspecto chave, pois quanto maior a taxa de compressão (menos componentes principais utilizados), maior a degradação da qualidade da imagem recuperada.

Em resumo, o autor ressaltou a importância do \acrshort{PCA} como uma abordagem eficaz para a compressão de imagens médicas, proporcionando economia de espaço de armazenamento sem comprometer significativamente a qualidade das imagens. A consideração da quantidade de componentes principais na compressão foi destacada como um aspecto crucial para garantir a recuperação adequada das informações visuais essenciais nas imagens médicas comprimidas.

\section{Compressão de imagens com transmissão em tempo real}
No estudo de \citeonline{sousa2015compressaoTempoReal}, foi adotada a compressão com perdas, onde a qualidade visual da imagem pode ser comprometida em troca de uma maior eficiência na transmissão. Para minimizar o custo computacional exigido, foi escolhida a Transformada de Wavelet discreta de Haar \cite{sousa2015compressaoTempoReal}, que, apesar de não proporcionar a melhor qualidade de imagem, é mais eficiente em termos de processamento, sendo adequada para aplicações em tempo real. A utilização da Transformada de Wavelet de Haar aliada à quantização permitiu a redução da quantidade de dados redundantes na imagem, garantindo uma compressão eficaz. O estudo buscou atender a uma taxa de transmissão de 30 FPS (\textit{frames per second}), o que exigiu a realização da compressão e transmissão de cada imagem em menos de 33,333 milissegundos para assegurar a transmissão em tempo real.

% Portanto, reconhecemos a importância desse trabalho, que se concentrou na busca por métodos eficientes que permitissem a transmissão de imagens em tempo real. Considerando a necessidade de equilibrar a qualidade visual com a velocidade de transmissão, esse estudo oferece contribuições valiosas para aplicações que requerem resposta em tempo real, como em sistemas de monitoramento e diagnóstico médico.

\section{Projeto de Arquiteturas Integradas para a compressão de imagens \acrshort{JPEG}}
O trabalho de \citeonline{agostini2002JPEGArch} se concentra na compressão de imagens fotográficas digitais, especialmente utilizando o padrão \acrshort{JPEG}, amplamente reconhecido e utilizado para a compressão de imagens. A motivação principal é a redução dos dados de imagem para otimizar recursos de armazenamento e transmissão. No sistema de monitoramento de Porto Alegre, as imagens não comprimidas geram altos custos de armazenamento e transporte. Com a compressão \acrshort{JPEG}, o sistema poderia aumentar sua capacidade de armazenamento, reduzir custos operacionais e melhorar eficiência geral.

O autor desenvolve arquiteturas específicas para compressão de imagens em \textit{Grayscale} e \acrshort{RGB}, incluindo a conversão de imagens do espaço de cores \acrshort{RGB} para YCbCr. A solução proposta envolve o desenvolvimento de arquiteturas descritas em VHDL, direcionadas para síntese em FPGA's da Altera, permitindo um processamento mais rápido e eficiente. Para imagens em \textit{Grayscale}, a arquitetura pode processar uma imagem de $640x480$ pixels em 18.5ms, permitindo uma taxa de 54 imagens por segundo. Para imagens \acrshort{RGB}, a arquitetura pode processar uma imagem de $640x480$ pixels em 54.4ms, com uma taxa de 18.4 imagens por segundo e uma compressão estimada de 93\%.

Em resumo, o autor apresenta uma solução eficiente para a compressão de imagens \acrshort{JPEG}, com aplicações potenciais em sistemas de monitoramento de trânsito e outras áreas que demandam processamento rápido e eficiente de imagens. A pesquisa contribui significativamente para a área, pois sugere desenvolve uma forma mais eficiente para comprimir imagens tanto em \textit{Grayscale} quanto em \acrshort{RGB}.